\section{内存模型和原子}
如果我们想成为并行程序员,内存一致性并不是一个深奥的概念。 它帮助我们确保数据在我们需要的时候出现在我们需要的地方,并且它的值是我们所期望的。 本章揭示了我们需要掌握的关键知识,以确保我们的程序正确运行。 这个主题并不是 SYCL 独有的。

对于任何想要允许并发更新内存的程序员来说,对编程语言的内存(一致性)模型有基本的了解是必要的(无论这些更新是否来自同一内核、多个设备或两者中的多个工作项)。 无论内存如何分配都是如此,无论我们选择使用缓冲区还是USM分配,本章的内容对我们来说都同样重要。

在前面的章节中,我们重点关注简单内核的开发,其中工作项要么对完全独立的数据进行操作,要么使用可以使用语言和/或库功能直接表达的结构化通信模式共享数据。 当我们转向编写更复杂和更现实的内核时,我们可能会遇到工作项可能需要以不太结构化的方式进行通信的情况 - 了解内存模型如何与 SYCL 语言功能以及我们目标硬件的功能相关。 设计正确、可移植、高效的程序的必要前提。

C++ 的内存一致性模型足以编写完全在主机上执行的应用程序,但它被 SYCL 修改,以解决编程异构系统时可能出现的复杂性。 具体来说,我们需要能够

• 系统中哪些设备可以访问哪些类型的内存分配(缓冲区和 USM)的原因

• 通过使用屏障和原子来防止内核执行期间不安全的并发内存访问(数据竞争)

• 使用屏障、栅栏、原子、内存顺序和内存范围实现工作项之间的安全通信

• 使用屏障、栅栏、原子、内存顺序和内存范围,防止可能意外改变并行应用程序行为的优化,同时仍允许其他优化

内存模型是一个复杂的主题,但有一个很好的理由——处理器架构师关心的是让处理器和加速器尽可能高效地执行我们的代码! 我们在本章中努力分解这种复杂性并突出最关键的概念和语言特征。 本章使我们不仅了解内存模型的内部和外部,而且还了解并行编程的一个重要方面,而许多人不知道它的存在。 如果阅读此处的描述和示例代码后仍有问题,我们强烈建议访问本章末尾列出的网站或参考 C++ 和 SYCL 规范。

\subsection{内存模型中有什么?}
本节扩展了编程语言包含内存模型的动机,并介绍了并行程序员应该熟悉的一些核心概念:

• 数据竞争和同步

• 障碍物和栅栏

• 原子操作

• 内存排序

为了理解它们在 C++ 和 SYCL 中的表达和用法,需要从高层次上理解这些概念。 在并行编程(尤其是使用 C++)方面具有丰富经验的读者可能希望跳过。

\subsubsection{数据竞争和同步}

\subsubsection{障碍和栅栏}

\subsubsection{原子操作}

\subsubsection{内存排序}

\subsection{内存模型}

\subsubsection{memory\_order枚举类}

\subsubsection{memory\_scope 枚举类}

\subsubsection{查询设备能力}

\subsubsection{障碍和栅栏}

\subsubsection{SYCL 中的原子操作}

\subsubsection{将原子与缓冲区一起使用}

\subsubsection{将原子与统一共享内存结合使用}

\subsection{在现实生活中使用原子}

\subsubsection{计算直方图}

\subsubsection{实现设备范围的同步}

\subsection{概括}

\subsubsection{了解更多信息}
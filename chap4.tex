\section{表达并行性}
我们已经知道如何在设备上放置代码(第 2 章)和数据(第 3 章)——我们现在要做的就是决定如何处理它。 
为此,我们现在转而填补一些迄今为止我们方便地遗漏或掩盖的事情。 
本章标志着从简单的教学示例到现实世界并行代码的转变,并扩展了我们在前面的章节中随意展示的代码示例的细节。

用一种新的并行语言编写我们的第一个程序似乎是一项艰巨的任务,特别是如果我们是并行编程的新手。 语言规范不是为应用程序开发人员编写的,并且通常假设对术语有一定的熟悉; 它们不包含以下问题的答案:

• 为什么有不止一种方式来表达并行性?

• 我应该使用哪种表达并行性的方法?

• 关于执行模型我到底需要了解多少?

本章旨在解决这些问题以及更多问题。 我们介绍了数据并行内核的概念,使用工作代码示例讨论了不同内核形式的优点和缺点,并强调了内核执行模型的最重要方面。

\subsection{内核内的并行性}
近年来,并行内核作为表达数据并行性的强大手段而出现。 基于内核的方法的主要设计目标是跨各种设备的可移植性和高程序员生产力。 因此,内核通常不会被硬编码为与特定数量或配置的硬件资源(例如,核心、硬件线程、SIMD [单指令,多数据]指令)一起工作。 相反,内核根据抽象概念来描述并行性,然后实现(即编译器和运行时的组合)可以将其映射到特定目标设备上可用的硬件并行性。 尽管此映射是实现定义的,但我们可以(并且应该)相信实现选择合理且能够有效利用硬件并行性的映射。

以与硬件无关的方式公开大量并行性可确保应用程序可以扩展(或缩小)以适应不同平台的功能,但是......

支持的设备存在很大的多样性,我们必须记住,不同的架构是针对不同的用例设计和优化的。 每当我们希望在特定设备上实现最高水平的性能时,无论我们使用哪种编程语言,我们都应该始终期望需要一些额外的手动优化工作! 此类特定于设备的优化的示例包括针对特定缓存大小的阻塞、选择分摊调度开销的工作粒度大小、利用专用指令或硬件单元,以及最重要的是选择适当的算法。 其中一些示例将在第 15、16 和 17 章中重新讨论。

在应用程序开发过程中在性能、可移植性和生产力之间取得适当的平衡是我们所有人都必须面对的挑战,也是本书无法完全解决的挑战。 然而,我们希望表明,带有 SYCL 的 C++ 提供了使用单一高级编程语言维护通用可移植代码和优化的目标特定代码所需的所有工具。 剩下的就留给读者作为练习了!

\subsection{循环与内核}
迭代循环本质上是串行构造:循环的每次迭代都是按顺序执行的(即按顺序)。 优化编译器也许能够确定循环的部分或全部迭代可以并行执行,但它必须是保守的 - 如果编译器不够智能或没有足够的信息来证明并行执行始终是安全的,则它 必须保留循环的顺序语义以确保正确性。

考虑图 4-1 中的循环,它描述了一个简单的向量加法。 即使在这样的简单情况下,证明循环可以并行执行也不是微不足道的:只有当 c 不与 a 或 b 重叠时,并行执行才是安全的,而在一般情况下,如果没有运行时检查,就无法证明这一点! 为了解决这样的情况,语言添加了一些功能,使我们能够为编译器提供额外的信息,这些信息可以简化分析(例如,断言指针不与限制重叠)或完全覆盖所有分析(例如,声明 循环是独立的或准确定义如何将循环调度到并行资源)。

并行循环的确切含义有些模糊(由于不同并行编程语言和运行时对该术语的重载),但许多常见的并行循环结构表示应用于顺序循环的编译器转换。 这种编程模型使我们能够编写顺序循环,然后才提供有关如何安全地并行执行不同迭代的信息。 这些模型非常强大,与其他最先进的编译器优化集成良好,并极大地简化了并行编程,但并不总是鼓励我们在开发的早期阶段考虑并行性。

并行内核不是循环并且没有迭代。 相反,内核描述了单个操作,该操作可以多次实例化并应用于不同的输入数据; 当并行启动内核时,该操作的多个实例可能会同时执行。

图 4-2 显示了使用伪代码重写为内核的简单循环示例。 该内核中的并行机会是清晰明确的:内核可以由任意数量的实例并行执行,并且每个实例独立地应用于单独的数据块。 通过将此操作编写为内核,我们断言并行运行是安全的(并且理想情况下应该并行运行)。

简而言之,基于内核的编程不是一种将并行性改进到现有顺序代码中的方法,而是一种编写显式并行应用程序的方法。

\subsection{多维内核}
许多其他语言的并行结构是一维的,将工作直接映射到相应的一维硬件资源(例如,硬件线程的数量)。 SYCL 中的并行内核是一个比这更高级别的概念,它们的维度更能反映我们的代码通常试图解决的问题(在一维、二维或三维空间中)。

然而,我们必须记住,并行内核提供的多维索引为程序员提供了便利,可以在底层一维空间之上实现。 了解这种映射的行为方式可能是某些优化(例如,调整内存访问模式)的重要部分。

一个重要的考虑因素是哪个维度是连续的或单位步幅(即,多维空间中的哪些位置在一维映射中彼此相邻)。 SYCL 中与并行性相关的所有多维量都使用相同的约定:维度从 0 到 N-1 进行编号,其中维度 N-1 对应于连续维度。 无论多维数量被写为列表(例如,在构造函数中)或类支持多个下标运算符,此编号都从左到右应用(从左侧的维度 0 开始)。 此约定与标准 C++ 中多维数组的行为一致。

使用 SYCL 约定将二维空间映射到线性索引的示例如图 4-3 所示。 我们当然可以自由地打破这个约定并采用我们自己的线性化索引的方法,但必须小心行事——打破 SYCL 约定可能会对受益于 strideone 访问的设备产生负面的性能影响。

如果应用程序需要三个以上的维度,我们必须负责使用模算术或其他技术手动在多维和线性索引之间进行映射。

\subsection{语言特性概述}
一旦我们决定编写并行内核,我们必须决定要启动什么类型的内核以及如何在程序中表示它。 表达并行内核的方法有很多种,如果我们想掌握这门语言,我们需要熟悉每一种方法。

\subsubsection{将内核与主机代码分离}
我们有几种分离主机和设备代码的替代方法,可以在应用程序中混合和匹配这些代码:C++ lambda 表达式或函数对象、通过互操作性接口定义的内核(例如 OpenCL C 源字符串)或二进制文件。 其中一些选项已在第 2 章中介绍,其他选项将在第 10 章和第 20 章中详细介绍。

所有这些选项都共享表达并行性的基本概念。 为了保持一致性和简洁性,本章中的所有代码示例都使用 C++ lambda 表达式来表达内核。

\begin{remark}[Lambda 表达式不被认为是有害的]
	为了开始使用 sYCl,无需完全理解 C++ 规范中有关 lambda 表达式的所有内容 - 我们需要知道的是 lambda 表达式的主体代表内核,并且(按值)捕获的变量将是 作为参数传递给内核。

使用 lambda 表达式而不是更详细的机制来定义内核不会对性能产生影响。 支持 sYCl 的 C++ 编译器始终能够理解 lambda 表达式何时表示并行内核的主体,并可以相应地针对并行执行进行优化。

有关 C++ lambda 表达式的复习及其在 sYCl 中的使用说明,请参阅第 1 章。有关使用 lambda 表达式定义内核的更多具体细节,请参阅第 10 章。
\end{remark}

\subsection{不同形式的并行内核}
SYCL中有三种不同的内核形式,支持不同的执行模型和语法。 可以使用任何内核形式编写可移植内核,并且可以调整以任何形式编写的内核以在各种设备类型上实现高性能。 然而,有时我们可能希望使用特定的形式来使特定的并行算法更容易表达或利用其他无法访问的语言功能。

第一种形式用于基本数据并行内核,并提供编写内核的最温和的介绍。 对于基本内核,我们牺牲了对调度等低级功能的控制,以使内核的表达尽可能简单。 各个内核实例如何映射到硬件资源完全由实现控制,因此随着基本内核复杂性的增加,推断其性能变得越来越困难。

第二种形式扩展了基本内核以提供对低级性能调整功能的访问。 由于历史原因,第二种形式被称为 ND 范围(N 维范围)数据并行,最重要的是要记住,它使某些内核实例能够分组在一起,从而允许我们对数据局部性和数据局部性进行一些控制。 各个内核实例和用于执行它们的硬件资源之间的映射。

第三种形式提供了一种实验性的替代语法,用于使用类似于嵌套并行循环的语法来表达 ND 范围内核。 第三种形式称为分层数据并行,指的是用户源代码中出现的嵌套结构的层次结构。 编译器对此语法的支持仍然不成熟,并且许多 SYCL 实现不能像其他两种形式那样有效地实现分层数据并行内核。 语法也不完整,因为 SYCL 有许多与分层内核不兼容或无法访问的性能支持功能。 SYCL 中的分层并行性正在更新过程中,并且 SYCL 规范包含一条注释,建议新代码在该功能准备就绪之前不要使用分层并行性; 为了与本说明的精神保持一致,本书的其余部分仅教授基本的和 ND 范围的并行性。

在更详细地讨论了不同内核形式的特性后,我们将在本章末尾再次讨论如何在不同内核形式之间进行选择。

\subsection{基础数据并行内核}
并行内核的最基本形式适用于高度并行的操作(即可以完全独立且以任何顺序应用于每条数据的操作)。 通过使用这种形式,我们可以实现对工作安排的完全控制。 因此,它是描述性编程构造的一个示例——我们描述操作是极其并行的,并且所有调度决策都是由实现做出的。

基本数据并行内核以单程序、多数据 (SPMD) 风格编写——单个“程序”(内核)应用于多条数据。 请注意,由于数据相关的分支,此编程模型仍然允许内核的每个实例在代码中采用不同的路径。

SPMD 编程模型的最大优势之一是它允许将相同的“程序”映射到多个级别和类型的并行性,而无需我们的任何明确指示。 同一程序的实例可以通过管道传输、打包在一起并使用 SIMD 指令执行、分布在多个硬件线程上或三者的混合。

\subsubsection{了解基本数据并行内核}
基本并行内核的执行空间被称为它的执行范围,并且内核的每个实例被称为一个项目。 图 4-4 对此进行了示意性表示。

基本数据并行内核的执行模型非常简单:它允许完全并行执行,但不保证或要求它。 项目可以按任何顺序执行,包括在单个硬件线程上顺序执行(即,没有任何并行性)! 因此,假设所有项目都将并行执行的内核(例如,通过尝试同步项目)可能很容易导致程序在某些设备上挂起。

然而,为了保证正确性,我们必须始终在假设内核可以并行执行的情况下编写内核。 例如,我们有责任确保对内存的并发访问受到原子内存操作(参见第 19 章)的适当保护,以防止竞争条件。

\subsubsection{编写基本数据并行内核}
基本数据并行内核使用parallel\_for 函数表示。 图 4-5 显示了如何使用这个函数来表达向量加法,这是我们对“Hello, world!”的看法。 用于并行加速器编程。

该函数仅接受两个参数:第一个是范围(或整数),指定在每个维度中启动的项目数,第二个是要对该范围中的每个索引执行的内核函数。 有几个不同的类可以被接受作为内核函数的参数,并且应该使用哪个类取决于哪个类公开所需的功能 - 我们稍后将重新讨论这一点。

图 4-6 显示了使用该函数非常类似地表达矩阵加法,除了二维数据之外,它与向量加法(在数学上)相同。 这由内核反映出来——两个代码片段之间的唯一区别是所使用的 range 和 id 类的维度! 可以用这种方式编写代码,因为 SYCL 访问器可以通过多维 id 进行索引。 尽管看起来很奇怪,但它非常强大,使我们能够编写根据数据维度模板化的通用内核。

在C/C++中更常见的是使用多个索引和多个下标运算符来索引多维数据结构,并且访问器也支持这种显式索引。 当内核同时操作不同维度的数据时,或者当内核的内存访问模式比直接使用项目的 id 描述的更复杂时,以这种方式使用多个索引可以提高可读性。

例如,图 4-7 中的矩阵乘法内核必须提取索引的两个单独分量,以便能够描述两个矩阵的行和列之间的点积。 作者认为,一致使用多个下标运算符(例如,[j][k])比混合多种索引模式和构造二维 id 对象(例如 id(j,k))更具可读性,但这是 只是个人喜好问题。

本章其余部分的示例都使用多个下标运算符,以确保所访问的缓冲区的维数不存在歧义。

图 4-8 中的图表显示了矩阵乘法内核中的工作如何映射到各个项目。 请注意,项目数是根据输出范围的大小得出的,并且多个项目可以读取相同的输入值:每个项目通过顺序迭代 C 矩阵的(连续)行来计算 C 矩阵的单个值。 A 矩阵和 B 矩阵的一个(不连续)列。

\subsubsection{基本数据并行内核的详细信息}
基本数据并行内核的功能通过三个 C++ 类公开:range、id 和 item。 我们已经在前面的章节中多次看到过 range 和 id 类,但我们在这里以不同的焦点重新审视它们。

\textbf{Range 类}

范围表示一维、二维或三维范围。 范围的维度是模板参数,因此必须在编译时已知,但每个维度的大小是动态的,并在运行时传递给构造函数。 range 类的实例用于描述并行构造的执行范围和缓冲区的大小。

图 4-9 显示了范围类的简化定义,显示了构造函数和查询其范围的各种方法。

\textbf{id 类}

id 表示一维、二维或三维范围的索引。 id 的定义在许多方面与 range 相似:它的维数也必须在编译时已知,并且它可用于索引并行构造中内核的单个实例或缓冲区中的偏移量。

如图 4-10 中 id 类的简化定义所示,id 在概念上只不过是一个、两个或三个整数的容器。 我们可用的操作也非常简单:我们可以查询每个维度中索引的组成部分,并且可以执行简单的算术来计算新的索引。

虽然我们可以构造一个 id 来表示任意索引,但要获取与特定内核实例关联的 id,我们必须接受它(或包含它的项)作为内核函数的参数。 这个id(或者它的成员函数返回的值)必须被转发到我们想要查询索引的任何函数——目前没有任何自由函数可以在程序中的任意点查询索引,但这可以简化为 SYCL 的未来版本。

每个接受 id 的内核实例只知道它被分配计算的范围内的索引,而对范围本身一无所知。 如果我们希望内核实例知道它们自己的索引和范围,我们需要使用 item 类。

\textbf{item 类}

项代表内核函数的单个实例,封装了内核的执行范围和该范围内的实例索引(分别使用范围和 id)。 与 range 和 id 一样,它的维数必须在编译时已知。

图 4-11 给出了项目类的简化定义。 item 和 id 之间的主要区别在于 item 公开了额外的函数来查询执行范围的属性(例如,其大小)以及计算线性化索引的便利函数。 与 id 一样,获取与特定内核实例关联的项的唯一方法是将其作为内核函数的参数接受。

\subsection{显式 ND 范围内核}
并行内核的第二种形式用项目属于组的执行范围替换基本数据并行内核的平坦执行范围。 这种形式最适合我们想要在内核中表达某些局部性概念的情况。 为不同类型的组定义和保证不同的行为,使我们能够更深入地了解和/或控制如何将工作映射到特定的硬件平台。

因此,这些显式 ND 范围内核是更具规范性的并行构造的示例 - 我们规定了工作到每种类型组的映射,并且实现必须遵守该映射。 然而,它并不完全是规定性的,因为组本身可以按任何顺序执行,并且实现对于每种类型的组如何映射到硬件资源保留了一定的自由度。 这种规范性和描述性编程的结合使我们能够针对局部性设计和调整内核,而不会破坏其可移植性。

与基本数据并行内核一样,ND 范围内核以 SPMD 风格编写,其中所有工作项都执行应用于多个数据的相同内核“程序”。 主要区别在于每个程序实例都可以查询其在包含它的组中的位置,并且可以访问特定于每种类型的组的附加功能(请参见第 9 章)。

\subsubsection{了解显式 ND 范围并行内核}
ND范围内核的执行范围分为工作组、子组和工作项。 ND-range 表示总的执行范围,它被划分为统一大小的工作组(即,工作组大小必须在每个维度上精确地除以 ND-range 大小)。 每个工作组可以根据实施进一步划分为子组。 了解为工作项和每种类型的组定义的执行模型是编写正确且可移植的程序的重要组成部分。

图 4-12 显示了大小为 (8, 8, 8) 的 ND 范围分为 8 个大小为 (4, 4, 4) 的工作组的示例。 每个工作组包含 16 个一维子组,每组有 4 个工作项。 请特别注意维度的编号:子组始终是一维的,因此 ND 范围和工作组的维度 2 成为子组的维度 0。

从每种类型的组到硬件资源的精确映射是实现定义的,正是这种灵活性使得程序能够在各种硬件上执行。 例如,工作项可以完全顺序执行、由硬件线程和/或SIMD指令并行执行、或者甚至由专门为内核配置的硬件管道执行。

在本章中,我们仅关注 ND 范围执行模型在通用目标平台方面的语义保证,并且我们不会涵盖其到任何一个平台的映射。 有关 GPU、CPU 和 FPGA 的硬件映射和性能建议的详细信息,请分别参阅第 15、16 和 17 章。

\subsubsection{编写显式 ND 范围数据并行内核}
\textbf{工作项}

工作项代表核函数的各个实例。 在没有其他分组的情况下,工作项可以按任何顺序执行,并且不能相互通信或同步,除非通过对全局内存的原子内存操作(参见第 19 章)。

\textbf{工作组}

ND 范围内的工作项被组织成工作组。 工作组可以按任何顺序执行,不同工作组中的工作项不能相互通信,除非通过对全局内存的原子内存操作(参见第 19 章)。 然而,当使用某些构造时,工作组内的工作项具有一些调度保证,并且该局部性提供了一些附加功能:

1. 工作组中的工作项可以访问工作组本地内存,该内存可能会映射到某些设备上的专用快速内存(请参阅第 9 章)。

2. 工作组中的工作项可以使用工作组屏障进行同步,并使用工作组内存栅栏保证内存一致性(参见第 9 章)。

3. 工作组中的工作项可以访问组功能,提供通用通信例程(参见第 9 章)和组算法的实现,提供通用并行模式的实现,例如归约和扫描(参见第 14 章)。

工作组中工作项的数量通常在运行时为每个内核配置,因为最佳分组将取决于可用并行度(即 ND 范围的大小)和目标设备的属性 。 我们可以使用设备类的查询函数确定特定设备支持的每个工作组的最大工作项数(参见第 12 章),并且我们有责任确保每个内核请求的工作组大小 已验证。

工作组执行模型中有一些微妙之处值得强调。

首先,虽然工作组中的工作项被调度到单个计算单元,但是工作组的数量和计算单元的数量之间不需要有任何关系。 事实上,ND 范围内的工作组数量可能比给定设备可以同时执行的工作组数量大很多倍! 我们可能会尝试编写通过依赖非常聪明的设备特定调度来跨工作组同步的内核,但我们强烈建议不要这样做——这样的内核今天可能可以工作,但不能保证它们将来也能工作 实现,并且当移动到不同的设备时很可能会中断。

其次,虽然工作组中的工作项目被安排为可以相互合作,但它们不需要提供任何具体的前进进度保证——在障碍和集体之间顺序执行工作组内的工作项目是一种 有效实施。 仅当使用提供的屏障和集合函数执行时,同一工作组中的工作项之间的通信和同步才能保证安全,并且手工编码的同步例程可能会死锁。

\textbf{子组}

在许多现代硬件平台上,工作组中的工作项子集(称为子组)在附加调度保证的情况下执行。 例如,子组中的工作项可以由于编译器向量化而同时执行,和/或子组本身可以以强大的前进进度保证来执行,因为它们被映射到独立的硬件线程。

当使用单一平台时,很容易将关于这些执行模型的假设融入到我们的代码中,但这使得它们本质上不安全且不可移植——在不同编译器之间移动时,甚至在不同代硬件之间移动时,它们可能会崩溃。 同一个供应商!

将子组定义为语言的核心部分为我们提供了一种安全的替代方案,可以避免做出稍后可能被证明是特定于设备的假设。 利用子组功能还允许我们在低级别(即接近硬件)推理工作项的执行,并且是跨许多平台实现非常高的性能水平的关键。

与工作组一样,子组内的工作项可以同步、保证内存一致性或通过组函数和组算法执行常见的并行模式。 然而,子组没有工作组本地存储器的等价物(即,没有子组本地存储器)。 相反,子组中的工作项可以使用组算法的子集(俗称“洗牌”操作)直接交换数据,无需显式内存操作(第 9 章)。

子组的某些方面是由实现定义的,不在我们的控制范围内。 然而,对于给定的设备、内核和 ND 范围的组合,子组具有固定(一维)大小,我们可以使用内核类的查询函数来查询该大小(参见第 10 章和第 12 章) 。 默认情况下,每个子组的工作项数量也由实现选择 - 我们可以通过在编译时请求特定的子组大小来覆盖此行为,但必须确保我们请求的子组大小与设备兼容 。

与工作组一样,子组中的工作项不需要提供任何特定的前进进度保证 - 实现可以自由地顺序执行子组中的每个工作项,并且仅在工作项发生变化时才在工作项之间切换。 遇到子群集体函数。 然而,在某些设备上,工作组内的所有子组都保证最终执行(取得进展),这是多种生产者-消费者模式的基石。 这是当前实现定义的行为,因此如果我们希望内核保持可移植性,我们就不能依赖子组来取得进展。 我们期望 SYCL 的未来版本能够提供描述子组进度保证的设备查询。

当为特定设备编写内核时,工作项到子组的映射是已知的,并且我们的代码通常可以利用此映射的属性来提高性能。 然而,一个常见的错误是假设因为我们的代码可以在一台设备上运行,所以它也可以在所有设备上运行。 图 4-13 和 4-14 仅显示了将范围为 {4, 4} 的多维内核中的工作项映射到子组(最大子组大小为 8)时的两种可能性。 图 4-13 生成两个包含 8 个工作项的子组,而图 4-14 中的映射生成四个包含 4 个工作项的子组!

SYCL 当前不提供查询工作项如何映射到子组的方法,也不提供请求特定映射的机制。 使用子组编写可移植代码的最佳方法是使用一维工作组或多维工作组,其中最高编号的维度可被内核所需的子组大小整除。

\textbf{编写显式 ND 范围数据并行内核}

图 4-15 使用 ND 范围并行内核语法重新实现了我们之前看到的矩阵乘法内核,图 4-16 中的图表显示了该内核中的工作如何映射到每个工作组中的工作项。 以这种方式对工作项进行分组可确保访问的局部性,并有望提高缓存命中率:例如,图 4-16 中的工作组的本地范围为 (4, 4),包含 16 个工作项,但仅访问 4 个工作项 数据量是单个工作项的数据量的四倍——换句话说,我们从内存加载的每个值都可以重复使用四次。

到目前为止,我们的矩阵乘法示例依赖于硬件缓存来优化同一工作组中的工作项对 A 和 B 矩阵的重复访问。 此类硬件缓存在传统 CPU 架构上很常见,并且在 GPU 架构上变得越来越常见,但一些架构已经明确管理可以提供更高性能(例如,通过更低延迟)的“暂存器”内存。 ND 范围内核可以使用本地访问器来描述应放置在工作组本地内存中的分配,然后实现可以自由地将这些分配映射到特殊内存(如果存在)。 该工作组本地内存的使用将在第 9 章中介绍。

\subsubsection{显式 ND 范围数据并行内核的详细信息}
与基本数据并行内核相比,ND 范围数据并行内核使用不同的类:range 被 nd\_range 替换,item 被 nd\_item 替换。 还有两个新类,代表工作项可能所属的不同类型的组:与工作组相关的功能封装在组类中,与子组相关的功能封装在 sub\_group 类中。

\textbf{nd\_range 类}

nd\_range 使用 range 类的两个实例表示分组执行范围:一个表示全局执行范围,另一个表示每个工作组的本地执行范围。 图 4-17 给出了 nd\_range 类的简化定义。

可能有点令人惊讶的是 nd\_range 类根本没有提及子组:子组范围在构造过程中没有指定并且无法查询。 造成这一遗漏的原因有两个。 首先,子组是一个低级实现细节,对于许多内核来说可以忽略。 其次,有多种设备恰好支持一种有效的子组大小,并且在任何地方指定该大小将是不必要的冗长。 与子组相关的所有功能都封装在一个专用类中,稍后将讨论该类。

\textbf{nd\_item 类}

nd\_item 是项目的 ND 范围形式,再次封装了内核的执行范围以及该范围内项目的索引。 nd\_item 与 item 的不同之处在于如何查询和表示其在范围中的位置,如图 4-18 中简化的类定义所示。 例如,我们可以使用 get\_global\_id() 函数查询(全局)ND 范围中的项目索引,或者使用 get\_local\_id() 函数查询项目在其(本地)父工作组中的索引。

nd\_item 类还提供了用于获取描述项目所属组和子组的类句柄的函数。 这些类提供了用于查询 ND 范围中项目索引的替代接口。

\textbf{group 类}

组类封装了与工作组相关的所有功能,简化的定义如图4-19所示。

group 类提供的许多函数在 nd\_item 类中都有等效的函数:例如,调用 group.get\_group\_id() 相当于调用 item.get\_group\_id(),调用 group.get\_local\_range() 相当于调用 item.get\_local\_range()。 如果我们不使用任何群函数或算法,我们还应该使用群类吗? 直接使用 nd\_item 中的函数而不是创建中间组对象不是更简单吗? 这里有一个权衡:使用 group 需要我们编写稍微多一些的代码,但该代码可能更容易阅读。 例如,考虑图 4-20 中的代码片段:很明显,body 期望被组中的所有工作项调用,并且很明显,parallel\_for 主体中的 get\_local\_range() 返回的范围是 组的范围。 仅使用 nd\_item 可以很容易地编写相同的代码,但读者可能会更难理解。

组类启用的另一个强大选项是能够编写通过模板参数接受任何类型的组的通用组函数。 尽管 SYCL(尚未)定义正式的 Group“概念”(在 C++20 意义上),但 group 和 sub\_group 类公开了一个公共接口,允许使用 sycl::is\_group\_v 等特征来约束模板化 SYCL 函数。 如今,这种通用编码形式的主要优点是能够支持具有任意维数的工作组,以及允许函数的调用者决定该函数是否应该在工作项之间划分工作的能力。 工作组或子组中的工作项。 然而,SYCL 组接口被设计为可扩展的,我们期望在 SYCL 的未来版本中出现更多代表不同工作项分组的类。

\textbf{sub\_group 类}

sub\_group类封装了与子组相关的所有功能,简化的定义如图4-21所示。 与工作组不同,sub\_group 类是访问子组功能的唯一方法; 它的功能在 nd\_item 中没有重复。

请注意,有单独的函数用于查询当前子组中的工作项数以及工作组内任何子组中的最大工作项数。 这些是否不同以及如何不同取决于具体设备的子组实现方式,但其目的是反映编译器目标子组大小与运行时子组大小之间的任何差异。 例如,非常小的工作组可以包含比编译时子组大小更少的工作项,或者可以使用不同大小的子组来处理不能被子组大小整除的工作组和维度。

\subsection{将计算映射到工作项}
到目前为止,大多数代码示例都假设内核函数的每个实例对应于对单条数据的单个操作。 这是一种编写内核的简单方法,但这种一对一的映射不是由 SYCL 或任何内核形式决定的——我们始终可以完全控制数据(和计算)分配给各个工作项,并且 使该分配可参数化可以是提高性能可移植性的好方法。

\subsubsection{一对一映射}
当我们编写内核以实现工作到工作项的一对一映射时,这些内核必须始终使用范围或 nd\_range 启动,其大小与需要完成的工作量完全匹配。 这是编写内核的最明显的方法,在许多情况下,它工作得非常好——我们可以相信一个实现可以有效地将工作项映射到硬件。

然而,当调整系统和实现的特定组合的性能时,可能需要更加密切地关注低级调度行为。 工作组对计算资源的调度是实现定义的,并且可能是动态的(即,当计算资源完成一个工作组时,它执行的下一个工作组可能来自共享队列)。 动态调度对性能的影响并不是固定的,其重要性取决于内核函数每个实例的执行时间以及调度是在软件(例如在CPU上)还是硬件(例如在CPU上)中实现的因素。 图形处理器)。

\subsubsection{多对一映射}
另一种方法是编写具有工作到工作项的多对一映射的内核。 在这种情况下,范围的含义发生了微妙的变化:范围不再描述要完成的工作量,而是描述要使用的工人数量。 通过更改工人数量和分配给每个工人的工作量,我们可以微调工作分配以最大限度地提高性能。

编写这种形式的内核需要进行两处更改:

1. 内核必须接受一个描述工作总量的参数。

2. 内核必须包含一个将工作分配给工作项的循环。

图 4-22 给出了此类内核的一个简单示例。 请注意,内核内部的循环有一种稍微不寻常的形式 - 起始索引是工作项在全局范围内的索引,步幅是工作项的总数。 这种数据到工作项的循环调度确保循环的所有 N 次迭代都将由工作项执行,而且线性工作项访问连续的内存位置(以改进缓存局部性和矢量化行为)。 工作可以类似地跨组或各个组中的工作项进行分配,以进一步改善局部性。

这些工作分配模式很常见,我们预计 SYCL 的未来版本将引入语法糖来简化 ND 范围内核中工作分配的表达。

\subsection{选择内核形式}
在不同的内核形式之间进行选择很大程度上取决于个人喜好,并且很大程度上受到其他并行编程模型和语言的先前经验的影响。

选择特定内核形式的另一个主要原因是它是公开内核所需的某些功能的唯一形式。 不幸的是,在开发开始之前很难确定需要哪些功能,特别是当我们仍然不熟悉不同的内核形式及其与各种类的交互时。

我们根据自己的经验构建了两个指南来帮助我们驾驭这个复杂的空间。 这些指南应被视为初步建议,绝对不是为了取代我们自己的实验 - 在不同内核形式之间进行选择的最佳方法始终是花一些时间编写每个内核形式,以便了解哪种形式是最好的 适合我们的应用和开发风格。

第一个指南是图4-23所示的流程图,它根据以下条件选择内核形式:

1.我们是否有并行编程的经验

2. 无论我们是从头开始编写新代码还是移植用不同语言编写的现有并行程序

3. 我们的内核是否是高度并行的,或者在内核函数的不同实例之间重用数据

4. 我们是否在 SYCL 中编写新内核是为了最大限度地提高性能、提高代码的可移植性,还是因为它提供了比低级语言更高效的表达并行性的方法

第二个指南是向每个内核形式公开的功能集。 工作组、子组、组屏障、组本地内存、组函数(例如广播)和组算法(例如扫描、归约)仅适用于 ND-range 内核,因此我们应该更喜欢 NDrange 内核 在我们有兴趣表达复杂算法或微调性能的情况下。

随着语言的发展,每种内核形式可用的功能预计会发生变化,但我们预计基本趋势保持不变:基本数据并行内核不会公开局部感知功能,显式 ND 范围内核将公开所有性能支持功能 特征。

\subsection{概括}
本章介绍了使用 SYCL 在 C++ 中表达并行性的基础知识,并讨论了每种编写数据并行内核的方法的优点和缺点。

SYCL 提供对多种形式的并行性的支持,我们希望我们已经提供了足够的信息来帮助读者做好准备并开始编码!

我们只触及了表面,接下来将更深入地探讨本章中介绍的许多概念和类:第 9 章介绍了本地内存、屏障和通信例程的使用; 除了使用 lambda 表达式之外定义内核的不同方法将在第 10 章和第 20 章中讨论; 第 15、16 和 17 章探讨了 ND 范围执行模型到特定硬件的详细映射; 第 14 章介绍了使用 SYCL 表达常见并行模式的最佳实践。
\section{FPGA 编程}

基于内核的编程最初作为一种访问 GPU 的方式而流行。 由于它现已推广到多种类型的加速器,因此了解我们的编程风格如何影响代码到 FPGA 的映射也很重要。

大多数软件开发人员都不熟悉现场可编程门阵列 (FPGA),部分原因是大多数台式计算机除了典型的 CPU 和 GPU 之外不包含 FPGA。 但 FPGA 值得了解,因为它们在许多应用中具有优势。 我们需要问与其他加速器相同的问题,例如“我什么时候应该使用 FPGA?”、“我的应用程序的哪些部分应该卸载到 FPGA?”以及“如何编写性能良好的代码” 在 FPGA 上?”

本章为我们提供了开始回答这些问题的知识,至少让我们能够确定 FPGA 是否适合我们的应用,并了解通常使用哪些结构来实现性能。 本章是我们可以阅读供应商文档以填写特定产品和工具链的详细信息的起点。 我们首先概述程序如何映射到 FPGA 等空间架构,然后讨论使 FPGA 成为加速器的良好选择的一些属性,最后介绍用于实现性能的编程结构。

本章中的“如何思考 FPGA”部分适用于思考任何 FPGA。 SYCL 允许供应商指定 CPU 和 GPU 之外的设备,但没有具体说明如何支持 FPGA。 本章中描述的 FPGA 特定供应商支持目前是 DPC++ 独有的,即 FPGA 选择器和管道。 FPGA 选择器和管道是本章中使用的唯一 DPC++ 扩展。 希望供应商能够采用类似或兼容的方式来支持 FPGA,DPC++ 作为开源项目也鼓励这样做。

\subsection{性能注意事项}
与任何处理器或加速器一样,FPGA 器件因供应商不同、甚至不同代产品也不同; 因此,一种设备的最佳实践可能并不适用于其他设备的最佳实践。 本章中的建议可能会使许多 FPGA 设备受益,无论是现在还是将来,但是……

\subsection{如何看待 FPGA}

\subsubsection{管道并行性}

\subsubsection{内核消耗芯片“区域”}

\subsection{何时使用 FPGA}

\subsubsection{很多很多的工作}

\subsubsection{自定义操作或操作宽度}

\subsubsection{标量数据流}

\subsubsection{低延迟和丰富的连接性}

\subsubsection{定制内存系统}

\subsection{在 FPGA 上运行}

\subsubsection{编译时间}

\subsubsection{FPGA 仿真器}

\subsubsection{FPGA 硬件编译“提前”进行}

\subsection{为 FPGA 编写内核}

\subsubsection{暴露并行性}

\subsubsection{使用 ND 范围保持管道繁忙}

\subsubsection{管道不介意数据依赖性!}

\subsubsection{循环的空间管道实现}

\subsubsection{循环启动间隔}

\subsubsection{管道}

\subsubsection{定制内存系统}

\subsection{一些结束语}

\subsubsection{FPGA 构建模块}

\subsubsection{时钟频率}

\subsection{概括}
在本章中,我们介绍了编译器如何将算法映射到 FPGA 的空间架构。 我们还介绍了一些概念,这些概念可以帮助我们确定 FPGA 是否对我们的应用有用,并且可以帮助我们更快地启动和运行代码开发。 从这个起点开始,我们应该能够很好地浏览供应商编程和优化手册并开始编写 FPGA 代码! FPGA 提供的性能并支持无法很好地映射到其他加速器的应用程序,因此我们应该将它们放在我们心理工具箱的前面!